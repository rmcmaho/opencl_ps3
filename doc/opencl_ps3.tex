 %% bare_conf.tex
%% V1.3
%% 2007/01/11
%% by Michael Shell
%% See:
%% http://www.michaelshell.org/
%% for current contact information.
%%
%% This is a skeleton file demonstrating the use of IEEEtran.cls
%% (requires IEEEtran.cls version 1.7 or later) with an IEEE conference paper.
%%
%% Support sites:
%% http://www.michaelshell.org/tex/ieeetran/
%% http://www.ctan.org/tex-archive/macros/latex/contrib/IEEEtran/
%% and
%% http://www.ieee.org/

%%*************************************************************************
%% Legal Notice:
%% This code is offered as-is without any warranty either expressed or
%% implied; without even the implied warranty of MERCHANTABILITY or
%% FITNESS FOR A PARTICULAR PURPOSE! 
%% User assumes all risk.
%% In no event shall IEEE or any contributor to this code be liable for
%% any damages or losses, including, but not limited to, incidental,
%% consequential, or any other damages, resulting from the use or misuse
%% of any information contained here.
%%
%% All comments are the opinions of their respective authors and are not
%% necessarily endorsed by the IEEE.
%%
%% This work is distributed under the LaTeX Project Public License (LPPL)
%% ( http://www.latex-project.org/ ) version 1.3, and may be freely used,
%% distributed and modified. A copy of the LPPL, version 1.3, is included
%% in the base LaTeX documentation of all distributions of LaTeX released
%% 2003/12/01 or later.
%% Retain all contribution notices and credits.
%% ** Modified files should be clearly indicated as such, including  **
%% ** renaming them and changing author support contact information. **
%%
%% File list of work: IEEEtran.cls, IEEEtran_HOWTO.pdf, bare_adv.tex,
%%                    bare_conf.tex, bare_jrnl.tex, bare_jrnl_compsoc.tex
%%*************************************************************************

%
\documentclass[conference]{IEEEtran}
%\documentclass{IEEEconf}

  % \usepackage[pdftex]{graphicx}
  % declare the path(s) where your graphic files are
  % \graphicspath{{../pdf/}{../jpeg/}}
  % and their extensions so you won't have to specify these with
  % every instance of \includegraphics
  % \DeclareGraphicsExtensions{.pdf,.jpeg,.png}

\usepackage{listings}
%numbers=left,
%numbersep=0pt,
\lstset{language=C,
        basicstyle=\ttfamily \footnotesize}

%\usepackage{algorithmic}
\usepackage{verbatim}

\begin{document}

\title{OpenCL on the Playstation 3}

\author{\IEEEauthorblockN{Robbie McMahon
\IEEEauthorblockA{Emerging Technologies Laboratory\\
Department of Computer Science\\
Loyola University Chicago\\
Chicago, IL 60640\\
rmcmaho@luc.edu}\\
}
}

\maketitle

\begin{abstract}
Programming for the Playstation 3 (PS3) is notoriously difficult due, in part, to its Cell Broadband Engine (Cell/B.E.).
Code must be written using the Cell SDK, which is useless on any other platform. Yet the system is incredibly powerful.
The current fastest computer in the world, Roadrunner, utilizes the Cell/B.E. to great effect.
The PS3 has become popular as a cheap supercomputer, but programmers are reluctant to port their code or even write new code.
OpenCL (Open Computing Language) seeks to solve this problem by providing a framework for parallel programming on heterogeneous systems.
OpenCL allows code to run on any architecture or device that supports the framework.
IBM has said a Cell/B.E. implementation is "in the works", but there is no hint as to when it will be available.
I aim to implement atleast a minimal subset of OpenCL for the Cell/B.E., allowing basic programs using OpenCL to run on the PS3.

\end{abstract}

\section{Motivation}\label{sec:motivation}
\input{motivation.tex}

\section{Related Work}
\input{relatedwork.tex}

\section{Project Plan}
I plan to implement a minimal subset of the
OpenCL specification. This will allow
basic programs using OpenCL to run on the PS3.
I will implement the required functions for
the setup and execution of tasks. This process
involves seven key steps which can be seen in
Appendix D of the OpenCL specification~\cite{opencl}.
\begin{enumerate}
\item Create an OpenCL context
\item Create a command-queue
\item Allocate memory objects
\item Create kernel
  \begin{enumerate}
    \item Create program
    \item Build program
    \item Get kernel
  \end{enumerate}
\item Set work-item dimensions
\item Execute kernel
\end{enumerate}

The three most important steps will 
be the most difficult to implement: creating contexts,
allocating memory objects, and creating kernels.
A context is a data structure that holds all information associated with a
device such as command-queues, memory objects,
and kernels. Allocating memory objects will
be difficult because I must conform to
the OpenCL memory model. Creating kernels
is actually several steps and possibly involves compiling
source code from within an executable. 

\subsection{Deliverables and Milestones}
To achieve this, I will create header and source files which
implement the OpenCL functions using the Cell SDK.
I will also write a short program which uses the
implemented functions. The program will compute
the integral of a simple polynomial using a
Reimann sum. This computation is easily parallelized
and I will be using code from a previous project.

There will be several key milestones in the project.
Note that this is not necessarily the order in which they will be completed.
However, the first milestone will be completed first because
all other milestones rely on it.
\begin{enumerate}
  \item Define the data structures passed 
    between the various functions. 
  \item Implement the context related functions
  \item Implement the memory object related functions
  \item Implement the kernel related functions
  \item Implement the remaining required OpenCL functions
\end{enumerate}

Because of the limited amount of time allocated
for this project, certain functions may be nothing
more than dirty hacks. This will be avoided whenever
possible, but the final deliverable may require them
to function.

\begin{comment}
The ultimate goal is to fully support OpenCL on the PS3. 
However, this may not be possible in the limited
5 week time frame.
\end{comment}


\section{Project Significance}
As stated earlier, there are no announced plans from
IBM regarding a PS3 implementation of OpenCL. If successful,
this would not only be the first PS3 implementation but
the first available and functional implementation.
To my knowledge, there are, as yet, no other implementations
of OpenCL. The developer releases of Apple's ``Snow Leopard''
have an implementation~\cite{applepr}, but those are not publicly available.
NVIDIA's implementation is also still under development~\cite{nvidiapr}~\cite{nvidiademo}.
I believe I can do this because I have done prior
research on both the Cell/B.E. and high performance,
parallel computing. I plan to release this under 
GPLv3 so that in the event IBM releases an official implementation,
there will still be a free and open-source version.


\section{Relevance to Distributed Systems}
\input{relevance.tex}

%\section{Analysis and Conclusion}
%\input{analysis.tex}

% references section
\bibliographystyle{IEEEtran}
\bibliography{IEEEabrv,sources}

\end{document}


