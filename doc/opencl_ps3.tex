 %% bare_conf.tex
%% V1.3
%% 2007/01/11
%% by Michael Shell
%% See:
%% http://www.michaelshell.org/
%% for current contact information.
%%
%% This is a skeleton file demonstrating the use of IEEEtran.cls
%% (requires IEEEtran.cls version 1.7 or later) with an IEEE conference paper.
%%
%% Support sites:
%% http://www.michaelshell.org/tex/ieeetran/
%% http://www.ctan.org/tex-archive/macros/latex/contrib/IEEEtran/
%% and
%% http://www.ieee.org/

%%*************************************************************************
%% Legal Notice:
%% This code is offered as-is without any warranty either expressed or
%% implied; without even the implied warranty of MERCHANTABILITY or
%% FITNESS FOR A PARTICULAR PURPOSE! 
%% User assumes all risk.
%% In no event shall IEEE or any contributor to this code be liable for
%% any damages or losses, including, but not limited to, incidental,
%% consequential, or any other damages, resulting from the use or misuse
%% of any information contained here.
%%
%% All comments are the opinions of their respective authors and are not
%% necessarily endorsed by the IEEE.
%%
%% This work is distributed under the LaTeX Project Public License (LPPL)
%% ( http://www.latex-project.org/ ) version 1.3, and may be freely used,
%% distributed and modified. A copy of the LPPL, version 1.3, is included
%% in the base LaTeX documentation of all distributions of LaTeX released
%% 2003/12/01 or later.
%% Retain all contribution notices and credits.
%% ** Modified files should be clearly indicated as such, including  **
%% ** renaming them and changing author support contact information. **
%%
%% File list of work: IEEEtran.cls, IEEEtran_HOWTO.pdf, bare_adv.tex,
%%                    bare_conf.tex, bare_jrnl.tex, bare_jrnl_compsoc.tex
%%*************************************************************************

%
\documentclass[conference]{IEEEtran}
%\documentclass{IEEEconf}

  % \usepackage[pdftex]{graphicx}
  % declare the path(s) where your graphic files are
  % \graphicspath{{../pdf/}{../jpeg/}}
  % and their extensions so you won't have to specify these with
  % every instance of \includegraphics
  % \DeclareGraphicsExtensions{.pdf,.jpeg,.png}

\usepackage{listings}
%numbers=left,
%numbersep=0pt,
\lstset{language=C,
        basicstyle=\ttfamily \footnotesize}

%\usepackage{algorithmic}
\usepackage{verbatim}

\begin{document}

\title{OpenCL on the Playstation 3}

\author{\IEEEauthorblockN{Robbie McMahon
\IEEEauthorblockA{Emerging Technologies Laboratory\\
Department of Computer Science\\
Loyola University Chicago\\
Chicago, IL 60640\\
rmcmaho@luc.edu}\\
}
}

\maketitle

\begin{abstract}
Programming for the Playstation 3 (PS3) is notoriously difficult due, in part, to its Cell Broadband Engine (Cell/B.E.).
Code must be written using the Cell SDK, which is useless on any other platform. Yet the system is incredibly powerful.
The current fastest computer in the world, Roadrunner, utilizes the Cell/B.E. to great effect.
The PS3 has become popular as a cheap supercomputer, but programmers are reluctant to port their code or even write new code.
OpenCL (Open Computing Language) seeks to solve this problem by providing a framework for parallel programming on heterogeneous systems.
OpenCL allows code to run on any architecture or device that supports the framework.
IBM has said a Cell/B.E. implementation is "in the works", but there is no hint as to when it will be available.
I aim to implement atleast a minimal subset of OpenCL for the Cell/B.E., allowing basic programs using OpenCL to run on the PS3.

\end{abstract}

\section{Motivation}\label{sec:motivation}

Parallel computing has become more popular with the introduction of
multi-core processors and GPGPU(general-purpose computing on graphics processing units).
But compared to sequential programs, parallel programs are far more complex.
Race conditions, deadlocks, and synchronization are just some of the
many hurdles when writing parallel programs. Running the program on differing
architectures simultaneously only adds to the complexity.
Yet this new technology can supply programs with incredible performance gains.
For example, the first supercomputer to break 1 Pflop/s, the Roadrunner at
Los Alamos National Laboratory, used the Cell/B.E. as performance accelerators
~\cite{roadrunner}. Some programs exclusively use the Cell/B.E.
for all computations, such as the VPIC particle-in-cell code ~\cite{vpic}.

Achieving such performance is difficult, particularly when it is a
hybrid system such as Roadrunner. When the architecture varies from
one system to the next like desktop computers, achieving peak performance
can be impossible. For example, optimizing a program to take full advantage
of both the Cell/B.E. and the NVIDIA Tesla ~\cite{tesla} simultaneously
and separately would certainly more than double the amount of
work required. OpenCL(Open Computing Language)~\cite{opencl} hopes to change this.


OpenCL aims to treat GPUs and CPUs as peers by abstracting away the hardware
thereby enabling developers to focus on producing high quality software~\cite{munshi}.
The abstraction is high enough to hide the implementation details,
but close enough to give low-level access to underlying hardware ~\cite{ocloverview}.
It is also open and royalty free with many vendors and industry leaders
planning to implement it. Unfortunately, there are currently
no released implementations of OpenCL. It remains merely a specification.

The first announced implementation will be part of Apple's new
operating system ``Snow Leopard''~\cite{applepr}, but Apple
has not yet set a release date. NVIDIA is also developing an
implementation on top of their CUDA architecture~\cite{nvidiapr}~\cite{nvidiaopencl}.
At SIGGRAPH 2008, NVIDIA demonstrated an n-body simulation based
on an early non-released OpenCL API/driver interface~\cite{nvidiademo}.
While IBM is part of the OpenCL working group, there have been
no specific announcements that an implementation for the Cell/B.E.
is forthcoming. This has been my impetus.

Having studied and used the PS3 for some research,
I have first hand experience with its power and capabilities.
I believe more researchers would use the platform if developing
software for it were made easier. Implementing OpenCL for the PS3
would be a major step in this direction.





%% These are just some ramblings and tests.
\begin{comment}
One such device is IBM's Cell Broadband Engine, or Cell BE for short. The processor is most commonly used in the Playstation 3 game system. The system has become popular among re

Unfortunately, programming for such devices can often be quite difficult. In addition, many systems combine several of these devices together, but most programs will only take advantage of one. OpenCL(Open Computing Language) is a new standard for implementing parallel programming on heterogeneous systems. It aims to allow the development of high-performance programs which are portable and platform-independent. Several companies have already announced their plan to support this new standard including Apple, AMD, RapidMind, and NVIDIA. 

It will support both data- and task- parallel model ~\cite{munshi}.
\end{comment}


\section{Related Work}
Parallel programming abstraction is not new.
MPI~\cite{mpi} and OpenMP~\cite{openmp} are both
well known parallel APIs. They allow developers to
write parallel code with little knowledge of the
underlying hardware. Unfortunately, they do not
directly support GPUs. There has been effort
to give GPGPU support for OpenMP~\cite{1504194}.
There are also a number of GPGPU clusters using
MPI~\cite{1278846}~\cite{1049991}. But the implementations
are platform specific and rely on homogeneous systems.
CUDA~\cite{cuda} is also fairly well established.
It is used in a number of research projects~\cite{1513899}~\cite{1513905}~\cite{1413373}.
But again, CUDA relies on homogeneous systems and 
only utilizes NVIDIA graphics devices. AMD tried
producing a GPGPU API called ``Close to Metal''~\cite{closetometal},
but it failed to catch on and is now deprecated.

There have been previous attempts to provide
high-level parallel programming on heterogeneous systems.
OpenCL is certainly not the first. Agora~\cite{36180}
was an early attempt at producing a programming
language designed with heterogeneous systems in mind.
It unfortunately did not become popular with developers.
More recently, there has been research which bears
resemblance to OpenCL. In his paper, Chiang proposed implicit parallelism
in programming~\cite{1071558}~\cite{1061887}. He did not suggest
a new language, but merely extensions to established languages.
There is also the Java Parallel Processing Framework~\cite{jppf}.
It is a grid computing platform written for Java. It allows
your program to run on any platform that has a Java
implementation.



\section{Project Plan}
I plan to implement a minimal subset of the
OpenCL specification. This will allow
basic programs using OpenCL to run on the PS3.
I will implement the required functions for
the setup and execution of tasks. This process
involves seven key steps which can be seen in
Appendix D of the OpenCL specification~\cite{opencl}.
\begin{enumerate}
\item Create an OpenCL context
\item Create a command-queue
\item Allocate memory objects
\item Create kernel
  \begin{enumerate}
    \item Create program
    \item Build program
    \item Get kernel
  \end{enumerate}
\item Set work-item dimensions
\item Execute kernel
\end{enumerate}

The three most important steps will 
be the most difficult to implement: creating contexts,
allocating memory objects, and creating kernels.
A context is a data structure that holds all information associated with a
device such as command-queues, memory objects,
and kernels. Allocating memory objects will
be difficult because I must conform to
the OpenCL memory model. Creating kernels
is actually several steps and possibly involves compiling
source code from within an executable. 

\subsection{Deliverables and Milestones}
To achieve this, I will create header and source files which
implement the OpenCL functions using the Cell SDK.
I will also write a short program which uses the
implemented functions. The program will compute
the integral of a simple polynomial using a
Reimann sum. This computation is easily parallelized
and I will be using code from a previous project.

There will be several key milestones in the project.
Note that this is not necessarily the order in which they will be completed.
However, the first milestone will be completed first because
all other milestones rely on it.
\begin{enumerate}
  \item Define the data structures passed 
    between the various functions. 
  \item Implement the context related functions
  \item Implement the memory object related functions
  \item Implement the kernel related functions
  \item Implement the remaining required OpenCL functions
\end{enumerate}

Because of the limited amount of time allocated
for this project, certain functions may be nothing
more than dirty hacks. This will be avoided whenever
possible, but the final deliverable may require them
to function.

\begin{comment}
The ultimate goal is to fully support OpenCL on the PS3. 
However, this may not be possible in the limited
5 week time frame.
\end{comment}


\section{Results}
The project was satisfactorily completed. It is now possible to compile and run the example D.1 code from the OpenCL specfication ~\cite{opencl} with only slight modifications. The program has only a handful of function calls, but each function is quite complex and will be explained in turn.

First, an OpenCL context is created using clCreateContextFromType().
\begin{verbatim}
context =
        clCreateContextFromType
        (NULL, CL_DEVICE_TYPE_CPU,
                NULL, NULL, NULL);
\end{verbatim}

The context holds all information regarding a system including devices, command queues, and programs. In the example code, the context is created from the CPU device type. The implemented function currently assumes the program is running on a Cell/B.E. processor. A custom function is called to create the context and device structures. The device should be created using clGetDeviceIDs(), but that will be left for further work. All of the parameters in the data structures are allocated and initialized. The function returns a valid cl\_context upon successful completion.

After creating the context, the devices are extracted from the context. This is done using clGetContextInfo().
\begin{verbatim}
clGetContextInfo
        (context, CL_CONTEXT_DEVICES,
                0, NULL, &cb);
devices = malloc(cb);
clGetContextInfo
        (context, CL_CONTEXT_DEVICES,
                cb, devices, NULL);
\end{verbatim}

The first line retrieves the number of devices. The pointers are allocated, and the final line sets the pointers to the devices within the context. The function can used to retrieve many other parameters of the context. This function, and others like it, are the only supported way of retrieving information about OpenCL data structures.

Having created the devices, the command queue can be initialized.
\begin{verbatim}
cmd_queue =
        clCreateCommandQueue(context,
                devices[0], 0, NULL);
\end{verbatim}

The command queue handles all commands that are to be scheduled onto a device. Program execution, memory allocation, and memory read/writes are all scheduled through the command queue. Currently, the command queue is just a place holder. It does not actually do anything. Section~\ref{sec:furtherwork} explains more about the command queue.

OpenCL memory objects are created in the example code, but they are not used in this project. 
\begin{verbatim}
memobjs[0] = clCreateBuffer(context,
        CL_MEM_READ_ONLY |
        CL_MEM_COPY_HOST_PTR,
        sizeof(cl_float4) * n,
        srcA, &err);
\end{verbatim}
The Cell/B.E has a peculiar memory hierarchy. Each SPE has its own local memory separate from main memory. This memory can only be accessed using specialized commands. Memory must also be aligned on 128 byte boundaries. Given the limited amount of time for the project, this data structure was skipped. See Section~\ref{sec:furtherwork} about possible expansion.

The key difference between the code for this project and the example code is the creation of OpenCL programs. The example code compiles source code and creates the program from the resultant binary. The project code uses pre-built SPE ELF binaries.
\begin{verbatim}
const char *input = "hello_spe.elf";
int size = strlen(input);
program = clCreateProgramWithBinary
               (context, 1, devices, 
                &size, &input,
                NULL, &err);
\end{verbatim}
This function simply sets the data structure parameters and calls a Cell SDK function to open the binary.
\begin{verbatim}
char *name = *(program->program_binaries);
name[(*lengths)] = '\0';
program->program_elfs =
        spe_image_open(name);  
\end{verbatim}
To run this code on a different OpenCL compatible device, the code would have to be modified. This would be a simple change such as adding C preprocessor macros. This is currently the only non-portable section of the code.

After loading the binary, the OpenCL kernel is created and the parameters are set.
\begin{verbatim}
kernel = clCreateKernel
        (program, "hello_spe", NULL);
cl_ulong argp = 12345;
err = clSetKernelArg(kernel, 0,
        sizeof(cl_ulong), (void *) &argp);
cl_ulong envp = 67890;
err = clSetKernelArg(kernel, 1,
        sizeof(cl_ulong), (void *) &envp);
\end{verbatim}
The kernel is normally a function to be run on the device. The function is supposed to search the OpenCL program for a function named "hello\_spe" with the "\_\_kernel" keyword. With a pre-built binary, this is quite difficult and unnecessary to the project. Instead, it simply initializes the parameters of the data structure. Only two arguments are set because SPE functions can only support a maximum of two arguments.

The final section enqueues the kernel to be run a set number of times.
\begin{verbatim}
err = clEnqueueNDRangeKernel(cmd_queue,
        kernel, 1, NULL, global_work_size,
        local_work_size, 0, NULL, NULL);
\end{verbatim}
Again, the command queue does not perform any actions at this time. The function manually executes the kernel on the Cell/B.E. the indicated number times. This is done using pthreads and the Cell SDK spe\_context\_run() function. "global\_work\_size" is the total number of times the kernel will execute. "local\_work\_size" is the number of kernels each compute unit will handle at a time.

After all the threads have finished and joined, the data structures are released and memory is freed. The program is successful if not completely portable. The program compiles and runs. The number of global work items can be changed and the SPE image can be set to any self contained SPE program (no DMA transfer, input/output, etc).


\section{Further Work}\label{sec:furtherwork}
The command queue certainly needs work. It should probably be a separate thread that processes incoming commands. This requires some design work and some concerns about concurrency. The command queue also requires a data structure to hold incoming commands.

Memory objects are also in need of some improvement. It is possible to do it on the PS3. The Cell/B.E. memory hierarchy actually lends itself quite nicely to the OpenCL specification. However, the system is still peculiar and a number of key points need to be addressed during the implementation, most noticeably the 128 byte aligned boundaries and DMA transfers which do not occur until after the kernel has begun execution.

Implementing the clCreateProgramFromSource() function would not be exceedingly difficult, but it would require many checks and error catching. C does allow system calls to be made, such as "gcc -o myprogram", but every system is different. This could be fixed by requiring the programmer to set the compiler of choice, but the function would still need to set all options and capture output and errors.

OpenCL contexts and devices are created using custom functions with hackish implementations. Specifically, devices should be created using the clGetDeviceIDs() function. The program also assumes it is running on the Cell/B.E. More effort should be made to confirm the assumption.

There is impressive documentation at the beginning of most functions, but the internals are lacking in this respect. More explanation about the process would benefit future programmer, particularly since this is an open source project.


\section{Project Significance}
As stated earlier, there are no announced plans from
IBM regarding a PS3 implementation of OpenCL. If successful,
this would not only be the first PS3 implementation but
the first available and functional implementation.
To my knowledge, there are, as yet, no other implementations
of OpenCL. The developer releases of Apple's ``Snow Leopard''
have an implementation~\cite{applepr}, but those are not publicly available.
NVIDIA's implementation is also still under development~\cite{nvidiapr}~\cite{nvidiademo}.
I believe I can do this because I have done prior
research on both the Cell/B.E. and high performance,
parallel computing. I plan to release this under 
GPLv3 so that in the event IBM releases an official implementation,
there will still be a free and open-source version.


\section{Relevance to Distributed Systems}
This project deals primarily with parallelism transparency.
Tasks can be run on multiple devices and in parallel
with little involvement from the programmer and zero
involvement from the user. Without OpenCL, programmers
would have to explicitly control parallel tasks. They
would also have to write code designed for a specific
type of architecture, whether it be a CPU, GPU, or some
other processor. 

With OpenCL, programmers are freed from these constraints.
They do not have to worry so much about specifics of
parallelism and architecture. The underlying hardware
can change and their software will still operate, given
that the hardware supports OpenCL. Users would also not have to worry
about downloading different binaries. Instead of
a binary for CPUs, one for GPUs, and one for both,
the user can just get one binary which will automatically
detect any available OpenCL compatible devices.


%\section{Analysis and Conclusion}
%\input{analysis.tex}

% references section
\bibliographystyle{IEEEtran}
\bibliography{IEEEabrv,sources}

\end{document}


